% -*-coding: utf-8 -*-

% 论文版芯453.55pt×700.15pt,按照从 ftp://hitgs.hit.edu.cn/下载的封面(20061211)
% 设置的版芯,可能开题报告不需装订,所以版芯比博士论文大。
% 页码在版芯下边线之下隔行居中放置;
% 左右
\setlength{\textwidth}{453.55pt}      % 文本宽度
\setlength{\oddsidemargin}{0pt}   % 左边 3.25cm=0.71+2.54 % 左边距
\setlength{\evensidemargin}{0pt}
%  上下
\setlength{\topmargin}{-32pt}       % 3.3=2.54+0.76  页面 上下动 0.42
\setlength{\headheight}{20pt}      % 0.8  页眉高度
\setlength{\headsep}{10pt}         % 0.4
\setlength{\textheight}{700.15pt}     % 21.0  文本高度
\setlength{\footskip}{26pt}        %1.1  调整页脚
%%%%%%%%%%%%%%%%%%%%%%%%%%%%%%%%%%%%%%%%%%%%%%%%%%%%%%%%%%%
%允许公式换页显示,否则大型推导公式都在一页内,
%一页显示不下放到第二页,导致很大的空白空间,很不好看
\allowdisplaybreaks[4]

%%%%%%%%%%%%%%%%%%%%%%%%%%%%%%%%%%%%%%%%%%%%%%%%%%%%%%%%%%%
%下面这组命令使浮动对象的缺省值稍微宽松一点,从而防止幅度
%对象占据过多的文本页面,也可以防止在很大空白的浮动页上放置很小的图形。
\renewcommand{\topfraction}{0.9999999}
\renewcommand{\textfraction}{0.0000001}
\renewcommand{\floatpagefraction}{0.9999}

%%%%%%%%%%%%%%%%%%%%%%%%%%%%%%%%%%%%%%%%%%%%%%%%%%%%%%%%%%%
% 重定义一些正文相关标题
%%%%%%%%%%%%%%%%%%%%%%%%%%%%%%%%%%%%%%%%%%%%%%%%%%%%%%%%%%%
\theoremstyle{plain} \theorembodyfont{\song\rmfamily}
\theoremheaderfont{\hei\rmfamily} %\theoremseparator{:}
\newtheorem{definition}{ 定义 }[section]
\newtheorem{proposition}[definition]{命题}
\newtheorem{lemma}[definition]{引理 }
\newtheorem{theorem}{定理 }[section]
\newtheorem{axiom}{公理}
\newtheorem{corollary}[definition]{推论}
\newtheorem{exercise}[definition]{}

%%%%%%%%%%%%%%%%%%%%%%%%%%%%%%%%%%%%%%%%%%%%%%%%%%%%%%%%%%%%%%%%%%%
%解决原proof定理环境的两个问题:
%  1. proof 中的item缩进不对
%  2. proof 中的最后一个公式下出现一个黑方块。
%\theoremsymbol{$\blacksquare$}
%\newtheorem{proof}{\hei 证明}
\newenvironment{proof}{\noindent{\hei 证明:}}{\hfill $ \square $ \vskip 4mm}
\theoremsymbol{$\square$}
%\newtheorem{example}{\hei 例}

%%%%%%%%%%%%%%%%%%%%%%%%%%%%%%%%%%%%%%%%%%%%%%%%%%%%%%%%%%%
% 用于中文段落缩进 和正文版式
%%%%%%%%%%%%%%%%%%%%%%%%%%%%%%%%%%%%%%%%%%%%%%%%%%%%%%%%%%%
%\CJKcaption{GB_aloft}
\CJKcaption{gb_452}

\newlength \CJKtwospaces

%\def\CJKindent{
%    \settowidth\CJKtwospaces{\CJKchar{"0A1}{"0A1}\CJKchar{"0A1}{"0A1}}%
%    \parindent\CJKtwospaces
%}

%\CJKtilde  \CJKindent

%\setlength{\parindent}{26pt} %由于工大论文的每行的字距加大了,需要增加段首缩2pt

\renewcommand\contentsname{\hei 目~~~~录}

%%%%%%章节标题为“第1章”的形式
\renewcommand\chaptername{\CJKprechaptername~\thechapter~\CJKchaptername}
%\renewcommand\chaptername{}%空
%%%%%%%%%%%%%%%%%%%%%%%%%%%%%%%%%%%%%%%%%%%%%%%%%%
%定义段落章节的标题和目录项的格式
%%%%%%%%%%%%%%%%%%%%%%%%%%%%%%%%%%%%%%%%%%%%%%%%%%
\setcounter{secnumdepth}{4} \setcounter{tocdepth}{2}

\titleformat{\chapter}[hang]{\xiaoer\bf\filcenter\hei\sf\boldmath}{\xiaoer\chaptertitlename}{18pt}{\xiaoer}
\titlespacing{\chapter}{0pt}{8pt}{16pt}

%\titleformat{\section}[hang]{\hei\sf\xiaosan\boldmath}{\xiaosan\thesection}{0.5em}{}
%\titlespacing{\section}{0pt}{13pt}{13pt}

\titleformat{\section}[hang]{\sectionfont\xiaosan\boldmath}{\sectionfont\xiaosan\thesection}{0.5em}{} %\sf 
\titlespacing{\section}{0pt}{13pt}{13pt}
\makeatletter
\renewcommand\thesection{\@arabic \c@section} % 前面不带 thechapter
\makeatother

\titleformat{\subsection}[hang]{\sectionfont\sihao\boldmath}{\sectionfont\sihao\thesubsection}{0.5em}{} %\sf
\titlespacing{\subsection}{0pt}{8pt}{7pt}

\titleformat{\subsubsection}[runin]{\sectionfont\xiaosi\boldmath}{\thesubsubsection}{0.5em}{}[\;\;] %\sf
%\titleformat{\subsubsection}[hang]{\hei\sf\xiaosi}{\xiaosi\thesubsubsection}{0.5em}{}
\titlespacing{\subsubsection}{0pt}{3pt}{2pt}   % #2与上文间距 #3与下文间距

%控制中文目录,开题没有目录
%\dottedcontents{chapter}[3.4em]{\vspace{0.5em}\hspace{-3.4em}\hei \bf\boldmath}{0.0em}{5pt}
%\dottedcontents{section}[1.16cm]{}{1.8em}{5pt}
%\dottedcontents{subsection}[2.00cm]{}{2.7em}{5pt}
%\dottedcontents{subsubsection}[2.86cm]{}{3.4em}{5pt}

%%%%%%%%%%%%%%%%%%%%%%%%%%%%%%%%%%%%%%%%%%%%%%%%%%%%%%%
% 定义页眉和页脚 使用fancyhdr 宏包
%%%%%%%%%%%%%%%%%%%%%%%%%%%%%%%%%%%%%%%%%%%%%%%%%%%%%%%%
\newcommand{\makeheadrule}{%
\makebox[-3pt][l]{\rule[.7\baselineskip]{\headwidth}{0.4pt}}
\rule[0.85\baselineskip]{\headwidth}{2.25pt}\vskip-.8\baselineskip}
\makeatletter
\renewcommand{\headrule}{%
    {\if@fancyplain\let\headrulewidth\plainheadrulewidth\fi
     \makeheadrule}}

\pagestyle{fancyplain}

%去掉章节标题中的数字
%%不要注销这一行,否则页眉会变成:“第1章1  绪论”样式
\renewcommand{\chaptermark}[1]{\markboth{\chaptertitlename~~ \ #1}{}}
 \fancyhf{}

%在book文件类别下,\leftmark自动存录各章之章名,\rightmark记录节标题
%% 页眉字号 工大要求 小五
%根据单双面打印设置不同的页眉;
\ifoneortwoside
  \fancyhead[CO]{\song\xiaowu\rightmark} %[scale=0.9]
  \fancyhead[CE]{\song\xiaowu 哈尔滨工业大学\cxueke\cxuewei 学位论文开题报告}% 调节页眉与横线的距离
  \fancyfoot[C,C]{\xiaowu\thepage}
\else
  \fancyhead[CO]{\song\xiaowu 哈尔滨工业大学\cxueke\cxuewei 学位论文开题报告}
  \fancyhead[CE]{\song\xiaowu 哈尔滨工业大学\cxueke\cxuewei 学位论文开题报告}% 调节页眉与横线的距离
  \fancyfoot[C,C]{\xiaowu\thepage}
\fi

%%%%%%%%%%%%%%%%%%%%%%%%%%%%%%%%%%%%%%%%%%%%%%%%%%%%%%%%
% 设置行距和段落间垂直距离
%%%%%%%%%%%%%%%%%%%%%%%%%%%%%%%%%%%%%%%%%%%%%%%%%%%%%%%%
%\renewcommand{\CJKglue}{\hskip 0.3pt plus 0.08\baselineskip}%加大字间距,使每行33个字
%\setlength{\belowcaptionskip}{10pt}   % 加大标题和表格之间的距离
\setlength{\parskip}{3pt plus1pt minus1pt} % 段落之间的竖直距离
\renewcommand{\baselinestretch}{1.2}% 定义行距

%%%%%%%%%%%%%%%%%%%%%%%%%%%%%%%%%%%%%%%%%%%%%%%%%%%%%%%%
% 调整列表环境的垂直间距
%%%%%%%%%%%%%%%%%%%%%%%%%%%%%%%%%%%%%%%%%%%%%%%%%%%%%%%%
\setitemize{itemindent=38pt,leftmargin=0pt,itemsep=-0.4ex,listparindent=26pt,partopsep=0pt,parsep=0.5ex,topsep=-0.25ex}
\setenumerate{itemindent=38pt,leftmargin=0pt,itemsep=-0.4ex,listparindent=26pt,partopsep=0pt,parsep=0.5ex,topsep=-0.25ex}
\setdescription{itemindent=38pt,leftmargin=0pt,itemsep=-0.4ex,listparindent=26pt,partopsep=0pt,parsep=0.5ex,topsep=-0.25ex}

%%参考文献
\renewcommand\bibsection{\section*{\centerline{\refname}}\vspace{-6pt}} %居中,与下文间距
\renewcommand\@biblabel[1]{[#1]\hspace{0.5em}} %参考文献里标号两边的括号
\newcommand{\ucite}[1]{$^{\mbox{\scriptsize \cite{#1}}}$} % 增加 \ucite 命令使显示的引用为上标形式
\newcommand{\citeup}[1]{$^{\mbox{\scriptsize \cite{#1}}}$} % for WinEdt users

%%%%%%%%%%%%%%%%%%%%%%%%%%%%%%%%%%%%%%%%%%%%%%%%%%%%%%%%%%%
% 定制浮动图形和表格标题样式 %这里用ccaption完全代替了caption2的功能
\captionstyle{\centering}   %不同的图标题形式采用不同的命令
%\indentcaption{0pt}           %参见ccaption
\hangcaption
\captionnamefont{\song\rmfamily\wuhao\selectfont}
\captiontitlefont{\song\rmfamily\wuhao\selectfont}
\captiondelim{~} %~

%%%%%%%%%%%%%%%%%%%%%%%%%%%%%%%%%%%%%%%%%%%%%%%%%%%%%%%
% 定义题头格言的格式
% 用法 \begin{Aphorism}{author}
%         aphorism
%      \end{Aphorism}
\newsavebox{\AphorismAuthor}
\newenvironment{Aphorism}[1]
{\vspace{0.5cm}\begin{sloppypar} \slshape
\sbox{\AphorismAuthor}{#1}
\begin{quote}\small\itshape }
{\\ \hspace*{\fill}------\hspace{0.2cm} \usebox{\AphorismAuthor}
\end{quote}
\end{sloppypar}\vspace{0.5cm}}

%自定义一个空命令,用于注释掉文本中不需要的部分。
\newcommand{\comment}[1]{}

\renewcommand\contentsname{\hei 目~~~~录}
\renewcommand\listfigurename{\hei 插~~~~图}
\renewcommand\listtablename{\hei 表~~~~格}

%%%%%%将章标题中的中文数字(一、二、三)变为阿拉伯数字(1,2,3)
%\renewcommand\CJKthechapter{%\CJKnumber
%{\@arabic\c@chapter}}

%%%%%%不要拉大行距使得页面充满
\raggedbottom

% This is the flag for longer version
\newcommand{\longer}[2]{#1}
\newcommand{\ds}{\displaystyle}
% define graph scale
\def\gs{1.0}

%%%%%%%%%%%%%%%%%%%%%%%%%%%%%%%%%%%%%%%%%%%%%%%%%%%%%%%%%%%%%%%%%%%%%%
% 自定义项目列表标签及格式 \begin{hitlist} 列表项 \end{hitlist}
%%%%%%%%%%%%%%%%%%%%%%%%%%%%%%%%%%%%%%%%%%%%%%%%%%%%%%%%%%%%%%%%%%%%%%
\newcounter{hitctr} %自定义新计数器
\newenvironment{hitlist}{%%%%%定义新环境
\begin{list}{{\hei (\arabic{hitctr})}} %%标签格式
    {
     \usecounter{hitctr}
     \setlength{\leftmargin}{0cm}     %左边界
     \setlength{\parsep}{0ex}         %段落间距
     \setlength{\topsep}{0pt}         %列表到上下文的垂直距离
     \setlength{\itemsep}{0ex}        %标签间距
     \setlength{\labelsep}{0.3em}     %标号和列表项之间的距离,默认0.5em
     \setlength{\itemindent}{46pt}    %标签缩进量
     \setlength{\listparindent}{27pt} %段落缩进量
    }}
{\end{list}}%%%%%

%%%%%%%%%%%%%%%%%%%%%%%%%%%%%%%%%%%%%%%%%%%%%%%%%%%%%%%%%%%%%%%%%%%%%%
% 自定义项目列表标签及格式 \begin{publist} 列表项 \end{publist}
%%%%%%%%%%%%%%%%%%%%%%%%%%%%%%%%%%%%%%%%%%%%%%%%%%%%%%%%%%%%%%%%%%%%%%
\newcounter{pubctr} %自定义新计数器
\newenvironment{publist}{%%%%%定义新环境
\begin{list}{\arabic{pubctr}} %%标签格式
    {
     \usecounter{pubctr}
     \setlength{\leftmargin}{2em}     % 左边界 \leftmargin =\itemindent + \labelwidth + \labelsep
     \setlength{\itemindent}{0em}     % 标号缩进量
     \setlength{\labelwidth}{1em}     % 标号宽度
     \setlength{\labelsep}{1em}       % 标号和列表项之间的距离,默认0.5em
     \setlength{\rightmargin}{0em}    % 右边界
     \setlength{\topsep}{0ex}         % 列表到上下文的垂直距离
%     \setlength{\partopsep}{0ex}      % 列表是一个新的段落时,加的额外到上下文的距离
     \setlength{\parsep}{0ex}         % 段落间距
     \setlength{\itemsep}{0ex}        % 标签间距
     \setlength{\listparindent}{26pt} % 段落缩进量
    }}
{\end{list}}%%%%%

%%%%%%%%%%%%%%%%%%%%%%%%%%%%%%%%%%%%%%%%%%%%%%%%%%%%%%%%%%%%%%%%%%%%%%
% 默认字体
\renewcommand\normalsize{%
  \@setfontsize\normalsize{12.1pt}{13pt}
  \setlength\abovedisplayskip{10pt plus 2pt minus 2pt}
  \setlength\abovedisplayshortskip{8pt plus 2pt minus 2pt}
  \setlength\belowdisplayskip{\abovedisplayskip}
  \setlength\belowdisplayshortskip{\abovedisplayshortskip}
  \setlength\jot{8pt}
  \let\@listi\@listI}
\def\defaultfont{\renewcommand{\baselinestretch}{1.37}\normalsize\selectfont}

%%%%%%%%%%%%%%%%%%%%%%%%%%%%%%%%%%%%%%%%%%%%%%%%%%%%%%%%%%%%%%%%%%%%%%
% 封面、摘要、版权、致谢格式定义
\makeatletter
\def\caffil#1{\def\@caffil{#1}}\def\@caffil{}
\def\csubject#1{\def\@csubject{#1}}\def\@csubject{}
\def\csupervisor#1{\def\@csupervisor{#1}}\def\@csupervisor{}
\def\cassosupervisor#1{\def\@cassosupervisor{~ & {\textbf{副\hfill 导\hfill 师}}& \rule[-3pt]{201pt}{1.2pt}\hspace{-201pt}\centerline{#1}\\[16pt]}}\def\@cassosupervisor{}
\def\cauthor#1{\def\@cauthor{#1}}\def\@cauthor{}
\def\cbdate#1{\def\@cbdate{#1}}\def\@cbdate{} %入学时间
\def\cdate#1{\def\@cdate{#1}}\def\@cdate{}    %开题日期
\def\ctitle#1{\def\@ctitle{#1}}\def\@ctitle{}
\def\ctitlesec#1{\def\@ctitlesec{~ & ~ & \rule[-3pt]{201pt}{1.2pt}\hspace{-201pt}\centerline{#1}}}\def\@ctitlesec{}
%%%%%%%%%%%%%%%%%%%%%%%%%%%%%%%%%%%%%%%%%%%%%%%%%%%%%%%%%%%%%%%
% 定义封面
\ifxueweidoctor
\def\makecover{
    \setboolean{@twoside}{true}
    %%%%%%%%%%%%%封面一
    \thispagestyle{empty}
    \vspace*{42pt}
    \begin{center}
    {\kai\xiaoer \makebox[152pt][s]{\textbf{哈尔滨工业大学}}}
    \end{center}

    \vspace{16pt}
    \begin{center}
      {\song\erhao\makebox[258pt][c]{\textbf{ \xueweishort \hfill 士 \hfill 学 \hfill 位 \hfill 论 \hfill 文 \hfill 开 \hfill 题 \hfill 报 \hfill 告}}}
    \end{center}

    \vspace{64.8pt}
    {\song\sanhao
    \def\oldtabcolsep{\tabcolsep}
    \setlength{\tabcolsep}{0pt}
    \noindent\begin{tabular}{p{69pt}p{97pt}p{201pt}lll}\setlength{\tabcolsep}{-30pt}
    ~ & {\textbf{院\hfill (系)}}                                   & \rule[-3pt]{201pt}{1.2pt}\hspace{-201pt}\centerline{\@caffil}\\[16pt]
    ~ & {\textbf{学\hfill 科}}                                     & \rule[-3pt]{201pt}{1.2pt}\hspace{-201pt}\centerline{\@csubject}\\[16pt]
    ~ & {\textbf{导\hfill 师}}                                     & \rule[-3pt]{201pt}{1.2pt}\hspace{-201pt}\centerline{\@csupervisor}\\[16pt]
    \@cassosupervisor
    ~ & {\textbf{研\hfill 究\hfill 生}}                            & \rule[-3pt]{201pt}{1.2pt}\hspace{-201pt}\centerline{\@cauthor}\\[16pt]
    ~ & {\textbf{入\hfill 学\hfill 时\hfill 间}}                   & \rule[-3pt]{201pt}{1.2pt}\hspace{-201pt}\centerline{\@cbdate}\\[16pt]
    ~ & {\textbf{开\hskip-0.4pt\hfill 题\hskip-0.4pt\hfill 报\hskip-0.4pt\hfill 告\hskip-0.4pt\hfill 日\hskip-0.4pt\hfill 期}} & \rule[-3pt]{201pt}{1.2pt}\hspace{-201pt}\centerline{\@cdate}\\[16pt]
    ~ & {\textbf{论\hfill 文\hfill 题\hfill 目}}                   & \rule[-3pt]{201pt}{1.2pt}\hspace{-201pt}\centerline{\@ctitle}\\[16pt]
    \@ctitlesec
    \end{tabular}
    \def\tabcolsep{\oldtabcolsep}
    \vspace{154pt}
    \begin{center}
    \textbf{研究生院培养处}
    \end{center}
}

%%定义内封
\newpage
\thispagestyle{empty}
\vspace*{14pt}
\begin{center}
  {\hei\sanhao \makebox[85pt][s]{说\hfill 明}}
\end{center}
\vspace*{40pt}
    {\song\renewcommand\baselinestretch{1.27}
    \fontsize{10.5pt}{12.6pt}\selectfont
    \noindent 一、开题报告应包括下列主要内容:
    \begin{enumerate}[leftmargin=36pt,itemindent=-2pt,itemsep=0ex,listparindent=21.8pt,partopsep=0pt,parsep=0.5ex,topsep=-0ex]
    \item 课题来源及研究的目的和意义;
    \item 国内外在该方向的研究现状及分析(文献综述);
    \item 前期的理论研究与试验论证工作的结果;
    \item 学位论文的主要研究内容、实施方案及其可行性论证;
    \item 论文进度安排,预期达到的目标;
    \item 为完成课题已具备和所需的条件、外协计划及经费;
    \item 预计研究过程中可能遇到的困难、问题,以及解决的途径;
    \item 主要参考文献(应在50篇以上,其中外文资料不少于二分之一,
    参考文献中近五年内发表的文献一般不少于三分之一,且必须有近二年
    内发表的文献资料)。
    \end{enumerate}
    \noindent 二、开题报告字数应不少于1.5万字。

    \noindent 三、开题报告时间应最迟应于第四学期结束前完成。

    \noindent 四、若本次开题报告未通过,\hspace{-1pt}需在三个月内再次进行开题报告。\hspace{-1pt}第二次学位论文开题报告 % 
    仍未通过\\ \hspace*{20.6pt}者,将取消其学籍。 %

    \noindent 五、开题报告结束后,\hfill 评议小组要填写\hfill《\cxuewei 学位论文开题报告评议结果》\hfill 上报院\hfill(系)\hfill 研
    究生教\\\hspace*{20.6pt}学秘书备案。%\hspace*{20.5pt}

    \noindent 六、此表不够填写时,可另加附页。
    }
    \clearpage
    \ifoneortwoside
    \else
        \setboolean{@twoside}{false}
    \fi
}%\makecover
\fi

\ifxueweimaster
\def\makecover{
    \setboolean{@twoside}{true}
    %%%%%%%%%%%%%封面一
    \thispagestyle{empty}
    \vspace*{42pt}
    \begin{center}
    {\kai\xiaoer \makebox[152pt][s]{\textbf{哈尔滨工业大学}}}
    \end{center}

    \vspace{16pt}
    \begin{center}
      {\song\xiaoyi\makebox[258pt][c]{\textbf{\xueweishort\hfill 士\hfill学\hfill位\hfill论\hfill文\hfill开\hfill题\hfill报\hfill告}}}
    \end{center}
    \vspace{36pt}
    \noindent{\song\xiaoer\textbf{题\ 目:}\parbox[b]{400pt}{\textbf{\@ctitle}}}\\[64.8pt]
    {\song\sanhao
    \def\oldtabcolsep{\tabcolsep}
    \setlength{\tabcolsep}{0pt}
    \noindent\begin{tabular}{p{69pt}p{97pt}p{201pt}lll}\setlength{\tabcolsep}{-30pt}
    ~ & {\textbf{院\hfill (系)}}                                   & \rule[-3pt]{201pt}{1.2pt}\hspace{-201pt}\centerline{\@caffil}\\[16pt]
    ~ & {\textbf{学\hfill 科}}                                     & \rule[-3pt]{201pt}{1.2pt}\hspace{-201pt}\centerline{\@csubject}\\[16pt]
    ~ & {\textbf{导\hfill 师}}                                     & \rule[-3pt]{201pt}{1.2pt}\hspace{-201pt}\centerline{\@csupervisor}\\[16pt]
    \@cassosupervisor
    ~ & {\textbf{研\hfill 究\hfill 生}}                            & \rule[-3pt]{201pt}{1.2pt}\hspace{-201pt}\centerline{\@cauthor}\\[16pt]
    ~ & {\textbf{年\hfill 级}}                   & \rule[-3pt]{201pt}{1.2pt}\hspace{-201pt}\centerline{\@cbdate}\\[16pt]
    ~ & {\textbf{开\hskip-0.4pt\hfill 题\hskip-0.4pt\hfill 报\hskip-0.4pt\hfill 告\hskip-0.4pt\hfill 日\hskip-0.4pt\hfill 期}} & \rule[-3pt]{201pt}{1.2pt}\hspace{-201pt}\centerline{\@cdate}\\[16pt]
    \end{tabular}
    \def\tabcolsep{\oldtabcolsep}
    \vfill%\vspace{154pt}
    \begin{center}
    \textbf{研究生院培养处制}
    \end{center}
}

%%定义内封
\newpage
\thispagestyle{empty}
\vspace*{44pt}
\begin{center}
  {\hei\sanhao \makebox[85pt][s]{说\hfill 明}}
\end{center}
\vspace*{40pt}
    {\song\renewcommand\baselinestretch{1.27}
    \fontsize{10.5pt}{12.6pt}\selectfont
    \noindent 一、开题报告应包括下列主要内容:
    \begin{enumerate}[leftmargin=36pt,itemindent=-2pt,itemsep=0ex,listparindent=21.8pt,partopsep=0pt,parsep=0.5ex,topsep=-0ex]
    \item 课题来源及研究的目的和意义;
    \item 国内外在该方向的研究现状及分析;
    \item 主要研究内容;
    \item 研究方案及进度安排,预期达到的目标;
    \item 为完成课题已具备和所需的条件和经费;
    \item 预计研究过程中可能遇到的困难、问题,以及解决的措施;
    \item 主要参考文献。
    \end{enumerate}
    \noindent 二、对开题报告的要求:
   \begin{enumerate}[leftmargin=36pt,itemindent=-2pt,itemsep=0ex,listparindent=21.8pt,partopsep=0pt,parsep=0.5ex,topsep=-0ex]
    \item 开题报告的字数应在5000字以上;
    \item 阅读的主要参考文献应在20篇以上,其中外文资料应不少于三分之一。硕士研究生应在
    导师的指导下着重查阅近年内发表的中、外文期刊文章。本学科的基础和专业课教材一般不应
    列为参考资料。
    \end{enumerate}
    \noindent 三、开题报告时间应最迟不得超过第三学期的第三周末。

    \noindent 四、\hspace{-1.5pt}如硕士生首次开题报告未通过,\hspace{-1.5pt}需在一个月内再进行一次。\hspace{-1.5pt}若仍不通过,\hspace{-1.5pt}则停止硕士论文工作。

    \noindent 五、此表不够填写时,可另加附页。

    \noindent 六、开题报告进行后,此表同硕士学位论文开题报告评议结果存各系(院)研究生秘
    书书处,以备\\\hspace*{20.6pt}研究生院及所属学院进行检查。
    }
    \clearpage
    \ifoneortwoside
    \else
        \setboolean{@twoside}{false}
    \fi
}
\fi
\makeatother
